\documentclass{article}

% Language setting
% Replace `english' with e.g. `spanish' to change the document language
\usepackage[french]{babel}
\usepackage[T1]{fontenc}

% Set page size and margins
% Replace `letterpaper' with `a4paper' for UK/EU standard size
\usepackage[a4paper,top=2cm,bottom=2cm,left=3cm,right=3cm,marginparwidth=1.75cm]{geometry}

% Useful packages
\usepackage{amsmath}
\usepackage{graphicx}
\usepackage[colorlinks=true, allcolors=blue]{hyperref}

% Other packages
\usepackage[indent]{parskip}

\title{Mini-projet 1}
\author{Kataevskii Mikhail, Ahrend Laurentin-Wilhelm, Garcia Harlouchet Ivan, \\ Galliot Noémie, Duhautois Lucas}

\begin{document}
\maketitle

\section{Présentation}

Le verre est un matériau d'avenir, recyclable à l'infini, mais sa production reste encore très polluante, notamment à cause de l'eau utilisée pour refroidir les systèmes et l'énergie dépensée pour chauffer les fours qui servent à la fusion du verre. De plus, les minéraux présent dans la composition initiale du verre relâchent du $CO_2$ lors de leur fusion.

De nombreuses pistes pour pallier à ce problème sont envisagées par les fabriquants, comme augmenter la proportion de calcin (verre recyclé) dans les préparations initiales, ou encore passer de fours à gaz à des fours électriques; mais les progrès faits sont encore loin d'être satisfaisant.

Le but de ce mini-projet est donc d'étudier une manière de réduire les émissions lors de la fabrication du verre. Pour cela, nous nous sommes concentrés sur la température de fusion de ce dernier. En effet, abaisser la température de fusion du verre signifie abaisser l'énergie nécessaire à sa fabrication. Cela permettrait aux entreprises de réaliser des économies d'énergies tout en diminuant leurs émissions. Cependant, le verre ainsi obtenu doit répondre au cahier des charges des entreprises et ne pas perdre en propriétés telles que sa dureté, sa flexibilité ou encore son élasticité.

Afin de répondre à cette problématique, nous disposons d'une base de données \textit{Interglad} regroupant les compositions et les propriétés de plusieurs centaines de milliers de verres.

Notre objectif est de développer un algorithme génétique exploitant cette base de données afin de trouver une composition de verre qui permet de minimiser la température de fusion tout en conservant les propriétés exigées. En effet, si il existe de nombreux algorithme visant à prédire les propriétés du verre à partir de sa composition, il n'en existe pas qui permette de prédire une composition à partir de propriétés spécifiques.

La composition ainsi obtenue sera utilisée pour créer notre propre verre, sur lequel nous pourrons expérimenter afin de tester ses capacités.

\section{Utilisation d'un algorithme de Deep Learning}
\subsection{Les propriétés recherchées}

Nous disposons de plusieurs modèles entraînés sur la base de données \textit{Interglad} qui permettent de prédire les propriétés du verre à partir de sa composition. Nous cherchons à déterminer un verre dont les fractions molaires vérifient :

\begin{itemize}
    \item $0.3 < x_{Si} < 0.75$
    \item $0.1 < x_{Na_2O} < 0.29$
    \item $x_{CaO} < 0.35$
    \item $x_{Al_2O_3} < 0.23$
\end{itemize} 

Nous cherchons à obtenir un verre qui possède les propriétés suivantes :

\begin{itemize}
    \item $1200 ^{\circ} C < T_m < 1300 ^{\circ} C$
    \item $2300 \text{ kg} \cdot \text{m}^{-3} < \rho < 2800 \text{ kg} \cdot \text{m}^{-3}$
    \item $500 ^{\circ} C < T_g < 600 ^{\circ} C$
    \item $70 < E < 90$
\end{itemize}

Pour fabriquer le verre nous avons accès aux composants suivantes : $SiO_2$, $Al_2O_3$, $MgO$, $CaO$, $Na_2O$, $K_2O$, $ZnO$, $TiO_2$

\subsection{Optimisation à partir de composants de base}

Nous avons d'abord cherché une composition optimale en se limitant aux 3 composants de base qu'on retrouve dans la plupart de verres : $SiO_2$, $NaO_2$ et $CaO$. Pour cela nous avons modifié la fonction de tirage au sort pour se limiter aux mélanges de ces 3 éléments. Nous avons ensuite trié la liste de compositions obtenues afin de trouver celle qui correspond le mieux à nos exigences.

La composition obtenue a été utilisée pour la première fusion du verre.

\subsection{Elaboration d'un algorithme génétique}
%TODO

\section{Fonte d'un premier verre}
\subsection{Composition}

En analysant les bases de données fournies par nos encadrants et celles sur Interglad, nous avons déterminé une première composition de verre, à l'aide uniquement de $SiO_2$, de $CaO$ et de $Na_2O$, dont la température de fusion prédite est comprise entre 1200°C et 1300°C. Cette composition molaire est la suivante:

\begin{itemize}
    \item 65,4 \% de $SiO_2$
    \item 11,3 \% de $CaO$
    \item 23,3 \% de $Na_2O$
\end{itemize}

Cette composition est très proche de celle utilisée par les égyptiens il y a plus de 2500 ans, d’après les écrits de la bibliothèque du roi Assurbanipal (70 \% de $SiO_2$, 10 \% de $CaO$, 20\% de $Na_2O$), car le verre est connu depuis très longtemps et sa composition a peu varié. 

En effet, un verre est composé des trois éléments principaux:
\begin{itemize}
    \item un formateur : Le formateur de verre forme la structure du verre. Il s'agit ici de $SiO_2$.
    \item Un fondant : Les fondants permettent la réduction de la température de fusion du verre en rompant certaines liaisons entre les molécules du formateur. Il s'agit ici de $Na_2O$
    \item Un ou plusieurs modificateur(s) de réseaux : Les modificateurs de réseaux permettent d'améliorer les propriétés du verre telles que la durabilité chimique ou mécanique. Il s'agit ici de $CaO$.
\end{itemize}

Les trois composants utilisés pour cette première compositions sont les composants les plus classiques, du fait de leurs rôles respectifs dans la formation du verre. Il n'est donc pas étonnant de retrouver une composition similaire à celle utilisée il y a déjà plusieurs milliers d'années. 

Par la suite, nous tenterons d'élaborer un verre plus complexe, notamment à l'aide d'ajouts d'oxides de minéraux.

\subsection{Fusion du mélange}

Les proportions citées ci-dessus ont ensuite été converties en proportions massiques, afin de réaliser le mélange à l'aide d'une balance de précision. Après homogénéisation, le mélange est introduit dans un creuset, lui-même inséré dans un support, avant d'être mis dans un four électrique, de modèle RHF 14/15 (voir en annexe ses capacités). Le support du creuset a été taillé afin d'avoir la même taille que ce dernier et ainsi permettre une meilleure prise pour la pince à la sortie du four.

Le protocole de fusion du verre est donné en annexe. Pour des raisons techniques de taille du creuset, la fonte est réalisée en deux parties. Une première partie du mélange est fondue puis une deuxième partie du mélange est réintroduite dans le liquide, et la fusion se poursuit.

Le liquide obtenu est ensuite coulé dans un moule en acier rectangulaire. Le verre ainsi obtenu est laissé à refroidir à l'air libre.

Toutes les manipulations en contact direct avec le four à de très hautes températures ont été réalisées par Franck Pigeonneau et Cristophe Prady, à l'aide d'équipements de protections contre ces températures.

\subsection{Recuisson du verre}

Le verre obtenu a été refroidi à température ambiante. Cependant, ce refroidissement a donc été inhomogène, plus rapide à l'extérieur qu'à l'intérieur. Cela a induit des contraintes, fragilisant le verre.

Pour pallier à ce problème, le verre est donc recuit. Il est enfourné dans un four dont la température est celle de sa transition vitreuse, soit environ 500°C. Le four est éteint et laissé à refroidir toute la nuit avec le verre à l'intérieur.
Cette étape permet la suppression des contraintes.

\subsection{Analyse des propriétés du verre}

Le verre obtenu est ensuite analysé. Il est tout d'abord exposé à de la lumière polarisante afin de vérifier si les contraintes ont bien disparue.
[photos à insérer et commenter]


Il est ensuite observé au microscope optique, où l'on peut voir la présence de rares infondus. Ce passage au microscope optique nous a permis également d'observer les bulles présentes dans le verre et de les compter, afin d'éliminer cette source d'erreur lors de la mesure de la densité.
 [photos à insérer et à commenter]

Le verre a ensuite été découpé afin de pouvoir mesurer sa dureté de Vicker et sa densité. 

\section{Annexe mini-projet 1}
\subsection{Protocole de fusion du verre}

Lors de la fusion du verre, nous avons suivis le protocole suivant :

\begin{enumerate}
    \item Préparation de la composition
    \item Enfournement d'une première partie du mélange vitrifiable
    \item Montée en température du four à 900°C
    \item Palier de 30 minutes
    \item Montée en température du four à Tm en 60 minutes
    \item Maintien de la température pendant 1h
    \item Enfournement du reste de la composition
    \item Remise en température à Tm pour 1h
    \item Coulée du verre dans un moule
\end{enumerate}

\subsection{Capacités du four électrique utilisé pour la fusion}

Les informations ci-dessous ont été fournies par le fabriquant (Carbolite) sur son site internet.

\begin{tabular}{|c|c|}
\hline
Température maximale (°C) & 1400  \\
\hline
Volume (L) & 15 \\
\hline
Temps de chauffe (min) & 35 \\
\hline
dimension interne HxLxP (mm) & 220x220x310 \\
\hline
Dimension externe HxLxP (mm) & 810x690x780 \\
\hline
Puissance max (W) & 10 000 \\
\hline
Puissance de maintien à température (W) & 2900 \\
\hline
Poids (kg) & 125 \\
\hline
\end{tabular} 


\end{document}
