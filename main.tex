\documentclass{article}

% Language setting
% Replace `english' with e.g. `spanish' to change the document language
\usepackage[french]{babel}
\usepackage[T1]{fontenc}

% Set page size and margins
% Replace `letterpaper' with `a4paper' for UK/EU standard size
\usepackage[letterpaper,top=2cm,bottom=2cm,left=3cm,right=3cm,marginparwidth=1.75cm]{geometry}

% Useful packages
\usepackage{amsmath}
\usepackage{graphicx}
\usepackage[colorlinks=true, allcolors=blue]{hyperref}

\title{Mini-projet 1}
\author{Kataevskii Mikhail, Ahrend Laurentin-Wilhelm, Garcia Harlouchet Ivan, \\ Galliot Noémie, Duhautois Lucas}

\begin{document}
\maketitle

\section{Présentation}

Bla Bla Bla, le verre, c'est magnifique, etc.

\section{Utilisation d'un algorithme de Deep Learning}
\subsection{Introduction}

On dispose de plusieurs modèles entrainés sur la base de données INTERGLAD qui permettent de prédire les propriétés du verre à partir de sa composition. On cherche à déterminer le verre dont les fractions molaires vérifient :
\begin{itemize}
    \item $0.3 < x_{Si} < 0.75$
    \item $x_{CaO} < 0.35$
    \item $0.1 < x_{Na_2O} < 0.29$
    \item $x_{Al_2O_3} < 0.23$
\end{itemize}
On veut que le verre possède les propriétés suivantes :
\begin{itemize}
    \item $1200 ^{\circ} C < T_m < 1300 ^{\circ} C$
    \item $2300 \text{ kg} \cdot \text{m}^{-3} < \rho < 2800 \text{ kg} \cdot \text{m}^{-3}$
    \item $500 ^{\circ} C < T_g < 600 ^{\circ} C$
    \item $70 < E < 90$
\end{itemize}

\end{document}
