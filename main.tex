\documentclass{article}

% Language setting
% Replace `english' with e.g. `spanish' to change the document language
\usepackage[french]{babel}
\usepackage[T1]{fontenc}

% Set page size and margins
% Replace `letterpaper' with `a4paper' for UK/EU standard size
\usepackage[letterpaper,top=2cm,bottom=2cm,left=3cm,right=3cm,marginparwidth=1.75cm]{geometry}

% Useful packages
\usepackage{amsmath}
\usepackage{graphicx}
\usepackage[colorlinks=true, allcolors=blue]{hyperref}

\title{Mini-projet 1}
\author{Kataevskii Mikhail, Ahrend Laurentin-Wilhelm, Garcia Harlouchet Ivan, \\ Galliot Noémie, Duhautois Lucas}

\begin{document}
\maketitle

\section{Présentation}

Le verre est un matériau d'avenir, recyclable à l'infini, mais sa production reste encore très polluante, notamment à cause de l'eau utilisée pour refroidir les systèmes et l'énergie dépensée pour chauffer les fours qui servent à la fusion du verre. De plus, les minéraux présent dans la composition initale du verre relâchent du $CO_2$ lors de leur fusion.

De nombreuses pistes pour pallier à ce problème sont envisagée par les fabriquants, comme augmenter la proportion de calcin -verre recyclé- dans les préparations initales, ou encore passer de fours à gaz à des fours électriques; mais les progrès faits sont encore loin d'être satisfaisant. \\

Le but de ce mini-projet est donc de proposer une manière de réduire les émissions lors de la fabrication du verre. Pour cela, nous nous sommes concentrés sur la température de fusion de ce dernier. En effet, abaisser la température de fusion du verre signifie abaisser l'énergie nécessaire à sa fabrication. Cela permettrait aux entreprises de réaliser des économies d'énergies tout en diminuant leur émissions.
Cependant, le verre ainsi obtenue doit répondre au cahier des charges des entreprises et ne pas perdre en propriétés telles que sa dureté, sa flexibilité ou encore son élasticité.

Afin de répondre à cette problématique, nous disposons d'une base de données [Nom] regroupant les compositions et les propriétés de plusieurs centaines de milliers de verres.\\

L'objectif est donc de trouver à l'aide d'un algorithme génétique exploitant cette base données une composition de verre permettant de mnimiser la température de fusion du verre tout en conservant ses propriétés. La composition ainsi obtenue sera utilisée expérimentalement pour créer notre propre verre, sur lequel nous pourrons expérimenter afin de tester ses capacités.

\section{Utilisation d'un algorithme de Deep Learning}
\subsection{Introduction}

On dispose de plusieurs modèles entrainés sur la base de données INTERGLAD qui permettent de prédire les propriétés du verre à partir de sa composition. On cherche à déterminer le verre dont les fractions molaires vérifient :
\begin{itemize}
    \item $0.3 < x_{Si} < 0.75$
    \item $x_{CaO} < 0.35$
    \item $0.1 < x_{Na_2O} < 0.29$
    \item $x_{Al_2O_3} < 0.23$
\end{itemize} 

On veut que le verre possède les propriétés suivantes :
\begin{itemize}
    \item $1200 ^{\circ} C < T_m < 1300 ^{\circ} C$
    \item $2300 \text{ kg} \cdot \text{m}^{-3} < \rho < 2800 \text{ kg} \cdot \text{m}^{-3}$
    \item $500 ^{\circ} C < T_g < 600 ^{\circ} C$
    \item $70 < E < 90$
\end{itemize}

Pour fabriquer le verre on a accès aux composantes suivantes :
% cette liste est moche...
\begin{itemize}
    \item $SiO_2$
    \item $Al_2O_3$
    \item $MgO$
    \item $CaO$
    \item $Na_20$
    \item $K_2O$
    \item $ZnO$
    \item $TiO_2$
\end{itemize}

\subsection{Optimisation avec 4 composantes}

\section{Fonte d'un premier verre}

\subsection{Composition}

En analysant les bases de données fournies par nos encadrants et celles sur Interglad, nous avons déterminé une première composition de verre, à l'aide uniquement de $SiO_2$, de $CaO$ et de $Na_20$, dont la température de fusion prédite est comprise entre 1200°C et 1300°C. Cette composition molaire est la suivante:

\begin{itemize}
    \item 65,4 \% de $SiO_2$
    \item 11,3 \% de $CaO$
    \item 23,3 \% de $Na_20$
\end{itemize}

Cette composition est très proche de celle utilisée par les égyptiens il y a plus de 2500 ans, d’après les écrits de la bibliothèque du roi Assurbanipal (70 \% de $SiO_2$, 10 \% de $CaO$, 20\% de $Na_20$). En effet, le verre est connu depuis très longtemps et sa composition a peu varié. \\

Ces proportions ont ensuite été converties en proportion massique, afin de réaliser le mélange à l’aide d’une balance de précision. Après homogénéisation, le mélange est introduit dans un creuset, lui-même inséré dans un support, avant d’être mis dans un four électrique. Le support du creuset a été taillé dans du graphite afin d’avoir la même taille que ce dernier et ainsi permettre une meilleure prise pour la pince à la sortie du four.\\

Le protocole de fusion du verre est donné en annexe. Pour des raisons techniques de taille du creuset, la fonte est réalisée en deux parties. Une première partie du mélange est fondue puis une deuxième partie du mélange est réintroduite dans le liquide, et la fusion se poursuit.\\

Le liquide obtenu est ensuite coulé dans un moule en acier rectangulaire.\\

\section{Annexe mini-projet 1}
\subsection{Protocole de fusion du verre}

Lors de la fusion du verre, nous avons suivis le protole suivant: \\

\begin{itemize}
    \item Préparation de la composition
    \item Enfournement d'une première partie du mélange vitrifiable
    \item Montée en température du four à 900°C
    \item Palier de 30 minutes
    \item Montée en température du four à Tm en 30 minutes
    \item Maintien de la température pendant 2h
    \item Enfournement du reste de la composition
    \item Remise en température à Tm pour 3h
    \item Montée à Tm + 50 °C en 30 minutes
    \item Coulée du verre
\end{itemize}

\end{document}
